\chapter{  Character sums}

Corresponding html file: \texttt{../Articles/Art15.html}










\section{Explicit Polya-Vinogradov inequalities}



The main Theorem of \cite{Qiu*91}
implies the following result.

\par 
\begin{thm}{Theorem (1991)}

  For $\chi$ a primitive character to the modulus $q > 1$, we have
$
\left|\sum\limits_{a=M+1}^{M+N}\chi(a)\right|
\le
\frac{4}{\pi^2}\sqrt{q}\log q+0.38\sqrt{q}+\frac{0.637}{\sqrt{q}}
$.
\par 
 When $\chi$ is not especially primitive, but is still non-principal, we
  have
$
\left|\sum\limits_{a=M+1}^{M+N}\chi(a)\right|
\le
\frac{8\sqrt{6}}{3\pi^2}\sqrt{q}\log q+0.63\sqrt{q}+\frac{1.05}{\sqrt{q}}
$.
\end{thm}

This was improved later by
\cite{Bachman-Rachakonda*01} into
the following.
\par 
\begin{thm}{Theorem (2001)}

  For $\chi$ a non-principal character to the modulus $q > 1$, we have
$
\left|\sum\limits_{a=M+1}^{M+N}\chi(a)\right|
\le
\frac{1}{3\log 3}\sqrt{q}\log q+6.5\sqrt{q}
$. 
\end{thm}


These results are superseded by
\cite{Frolenkov*11} and more
recently by
\cite{Frolenkov-Soundararajan*13} into
the following.
\par 
\begin{thm}{Theorem (2013)}

  For $\chi$ a non-principal character to the modulus $q\ge 1000$, we have
$
\left|\sum\limits_{a=M+1}^{M+N}\chi(a)\right|
\le
\frac{1}{\pi\sqrt{2}}\sqrt{q}(\log q+6)+\sqrt{q}
$. 
\end{thm}


In the same paper they improve upon estimates of
\cite{Pomerance*11} and get the following.
\par 
\begin{thm}{Theorem (2013)}

  For $\chi$ a primitive character to the modulus $q \ge 1200$, we have
$$
\max_{M,N}\left|\sum_{a=M+1}^{M+N}\chi(a)\right|
\le
\begin{cases}
\frac{2}{\pi^2}\sqrt{q}\log q+\sqrt{q},&
  \text{$\chi$ even,}\\
\frac{1}{2\pi}\sqrt{q}\log q+\sqrt{q},&
  \text{$\chi$ odd}.
\end{cases}
$$
This latter estimates holds as soon as $q\ge40$.
\end{thm}


In case $\chi$ odd, the constant $1/(2\pi)$ has already
been asymptotically obtained in
\cite{Landau*18-3}.
When $\chi$ is odd and $M=1$, the best asymptotical constant before 2020 was
$1/(3\pi)$ from Theorem 7 of
\cite{Granville-Soundararajan*07},
In case $\chi$ even, we have
$$
\max_{M,N}\left|\sum_{a=M}^N\chi(a)\right|
=2\max_{N}\left|\sum_{a=1}^N\chi(a)\right|.
$$
(The LHS is always less than the RHS. Equality is then easily proved).
The asymptotical best constant in 2007
was $23/(35\pi\sqrt{3})$ from Theorem 7 of
\cite{Granville-Soundararajan*07}.

These results are improved upon for large values squarefree values of $q$ in
\cite{Bordignon-Kerr*20}
by a different method into the following.
\par 
\begin{thm}{Theorem (2020)}

  For $\chi$ a primitive character to the squarefree modulus $q \ge \exp(1088\ell^2)$, we have
$$
\max_{N}\left|\sum_{a=1}^{N}\chi(a)\right|
\le
\begin{cases}
  \frac{2}{\pi^2}\sqrt{q}\bigl(\frac14+\frac{1}{4\ell}\bigr)\log q
  +\bigl(49+\frac{1}{1088\ell}\bigr)\sqrt{q},&
  \text{$\chi$ even,}\\
  \frac{1}{2\pi}\bigl(\frac12+\frac{1}{2\ell}\bigr)\sqrt{q}\log q
  +\bigl(49+\frac{1}{1088\ell}\bigr)\sqrt{q},&
  \text{$\chi$ odd}.
\end{cases}
$$
This latter estimates holds as soon as $q\ge40$.
\end{thm}


Corresponding estimates when $q$ is not squarefree are proved in
\cite{Bordignon*21}, the
saving $1/4$ being slightly degraded to $3/8$.



\section{Burgess type estimates}


The following from 
\cite{Trevino*15-2}
is an explicit version of Burgess with the only restriction being
$p\ge 10^7$.
\par 
\begin{thm}{Theorem (2015)}

Let $p$ be a prime such that $p \ge 10^7$. Let $\chi$ be a non-principal character $\bmod{\,p}$. Let $r$ be a positive integer, and let $M$ and $N$ be non-negative integers with $N\ge 1$. Then
$$
\left|\sum_{a=M+1}^{M+N}\chi(a)\right|
\le 2.74 N^{1-\frac{1}{r}}
p^{\frac{r+1}{4r^2}}(\log{p})^{\frac{1}{r}}.
$$
\end{thm}

\par 
  From the same paper, we get the following more specific result.
\par 
\begin{thm}{Theorem (2015)}

Let $p$ be a prime. Let $\chi$ be a non-principal character
$\bmod{\,p}$. Let $M$ and $N$ be non-negative integers with $N\ge 1$,
let $2\le r\le 10$ be a positive integer, and let $p_0$ be a positive
real number. Then for $p \ge p_0$, there exists $c_1(r)$, a constant
depending on $r$ and $p_0$ such that 
$$
\left|\sum_{a=M+1}^{M+N}\chi(a)\right|
\le
c_1(r) N^{1-\frac{1}{r}} p^{\frac{r+1}{4r^2}}(\log{p})^{\frac{1}{r}}
$$
where $c_1(r)$ is given by

  
  
    
      $r$
      $p_0=10^7$
      $p_0=10^{10}$
      $p_0=10^{20}$
    
  
  
    2
    2.7381
    2.5173
    2.3549
  
  
    3
    2.0197
    1.7385
    1.3695
  
  
    4
    1.7308
    1.5151
    1.3104
  
  
    5
    1.6107
    1.4572
    1.2987
  
  
    6
    1.5482
    1.4274
    1.2901
  
  
    7
    1.5052
    1.4042
    1.2813
  
  
    8
    1.4703
    1.3846
    1.2729
  
  
    9
    1.4411
    1.3662
    1.2641
  
  
    10
    1.4160
    1.3495
    1.2562
  


\end{thm}

\par 

 We can get  a smaller exponent on $\log$ if we restrict the range of
 $N$ or if we have $r\ge 3$.
\par 
\begin{thm}{Theorem (2015)}

Let $p$ be a prime. Let $\chi$ be a non-principal character
$\bmod{\,p}$. Let $M$ and $N$ be non-negative integers with $1\le N\le
2 p^{\frac{1}{2} + \frac{1}{4r}}$ or $r\ge 3$. Let $r\le 10$ be a
positive integer, and let $p_0$ be a positive real number. Then for $p
\ge p_0$, there exists $c_2(r)$, a constant depending on $r$ and $p_0$
such that 
$$
\left|\sum_{a=M+1}^{M+N}\chi(a)\right|
\le
 c_2(r) N^{1-\frac{1}{r}} p^{\frac{r+1}{4r^2}}(\log{p})^{\frac{1}{2r}},
$$
where $c_2(r)$ is given by
  
  
  
    
      $r$
      $p_0=10^7$
      $p_0=10^{10}$
      $p_0=10^{20}$
    
  
  
    2
    3.7451
    3.5700
    3.5341
  
  
    3
    2.7436
    2.5814
    2.4936
  
  
    4
    2.3200
    2.1901
    2.1071
  
  
    5
    2.0881
    1.9831
    1.9037
  
  
    6
    1.9373
    1.8504
    1.7748
  
  
    7
    1.8293
    1.7559
    1.6843
  
  
    8
    1.7461
    1.6836
    1.6167
  
  
    9
    1.6802
    1.6262
    1.5638
  
  
    10
    1.6260
    1.5786
    1.5210
  
 

\end{thm}

\par 

Kevin McGown  in
\cite{McGown*12}
has slightly worse constants in a slightly larger range of $N$ for
smaller values of $p$.
\par 
\begin{thm}{Theorem (2012)}

Let $p\ge 2\cdot 10^{4}$ be a prime number. Let $M$ and $N$ be
non-negative integers with $1\le N\le 4 p^{\frac{1}{2} +
\frac{1}{4r}}$. Suppose $\chi$ is a non-principal character
$\bmod{\,p}$. Then there exists a computable constant $C(r)$ such that
$$
\left|\sum_{a=M+1}^{M+N}\chi(a)\right|
\le
C(r) N^{1-\frac{1}{r}} p^{\frac{r+1}{4r^2}}(\log{p})^{\frac{1}{2r}},
$$
where $C(r)$ is given by

  
    
      
	$r$
	$C(r)$
	$r$
	$C(r)$
      
    
    
      2
      10.0366
      9
      2.1467
    
    
      3
       4.9539
      10
      2.0492
    
    
      4
      3.6493
      11
      1.9712
    
    
      5
      3.0356
      12
      1.9073
    
    
      6
      2.6765
      13
      1.8540
    
    
      7
      2.4400
      14
      1.8088
    
    
      8
      2.2721
      15
      1.7700
    
  

\end{thm}

\par 


If the character is quadratic (and with a more restrictive
range), we have slightly stronger results due to Booker in  
\cite{Booker*06}.
\par 
\begin{thm}{Theorem (2006)}

Let $p > 10^{20}$ be a prime number with $p \equiv 1 \pmod{4}$. Let
$r\in \{2,3,4,\ldots,15\}$. Let $M$ and $N$ be real numbers such that
$0 < M , N \le 2\sqrt{p}$. Let $\chi$ be a non-principal quadratic
character $\bmod{\,p}$.  Then
$$
\left|\sum_{a=M+1}^{M+N}\chi(a)\right|
\le \alpha(r) N^{1-\frac{1}{r}} p^{\frac{r+1}{4r^2}}\left(\log{p} +
     \beta(r)\right)^{\frac{1}{2r}},
$$
     where $\alpha(r)$ and $\beta(r)$ are given by

  
  
    
      $r$
      $\alpha(r)$
      $\beta(r)$
      $r$
      $\alpha(r)$
      $\beta(r)$
    
  
  
    2
    1.8221
    8.9077
    9
    1.4548
    0.0085
  
  
    3
    1.8000
    5.3948
    10
    1.4231
    -0.4106
  
  
    4
    1.7263
    3.6658
    11
    1.3958
    -0.7848
  
  
    5
    1.6526
    2.5405
    12
    1.3721
    -1.1232
  
  
    6
    1.5892
    1.7059
    13
    1.3512
     -1.4323
  
  
    7
    1.5363
    1.0405
    14
    1.3328
    -1.7169
  
  
    8
     1.4921
    0.4856
    15
    1.3164
    -1.9808
  
  

\end{thm}


Concerning composite moduli, we have the next result in
\cite{Jain-Sharma-Khale-Liu*21}
.
\par 
\begin{thm}{Theorem (2021)}

  Let $\chi$ be a primitive character with modulus $q\ge e^{e^{9.594}}$.
  Then for $N\le q^{5/8}$, we have
$$
\left|\sum_{a=M+1}^{M+N}\chi(a)\right|
  \le 9.07 \sqrt{N}q^{3/16}(\log q)^{1/4}
  \bigl(2^{\omega(q)}d(q)\bigr)^{3/4}
  \sqrt{\frac{q}{\varphi(q)}}.
$$
\end{thm}

\par 


 
 








  
\begin{flushright}\small\sl{}   Last updated on December 11th, 2021, by Olivier Ramar\'e
 \end{flushright}















