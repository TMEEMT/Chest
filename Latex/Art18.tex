\chapter{  Bounds on the Dedekind zeta-function}

Corresponding html file: \texttt{../Articles/Art18.html}










 
 


\par 
\section{Size }


The knowledge on the general Dedekind zeta is less accomplished than
the one of the Riemann zeta-function, but we still have interesting
results. Theorem 4 of \cite{Rademacher*59} gives
the convexity bound. See also section 4.1 of
\cite{Trudgian*13}.
\par 
\begin{thm}{Theorem (1959)}

In the strip $-\eta\le \sigma\le 1+\eta$, $0 < \eta\le 1/2$, the Dedekind zeta
function $\zeta_K(s)$ belonging to the algebraic number field $K$ of degree
$n$ and discriminant $d$ satisfies the inequality
$$
|\zeta_K(s)|\le 3 \left|\frac{1+s}{1-s}\right|
\left(\frac{|d||1+s|}{2\pi}\right)^{\frac{1+\eta-\sigma}{2}}
\zeta(1+\eta)^n.
$$
\end{thm}


\par 

\section{Number of ideals}


An explicit approximation of the number of ideals in a number field
was given in the PhD memoir
\cite{Sunley*73}
of J.S. Sunley. It is recalled in Theorem 1.1 of
\cite{Lee*22}
and further refined there in Theorem 1.2.
\par 
\begin{thm}{Theorem (2022)}

    For $x > 0$ and $n_K\ge 0$, the number $I_K(x)$ of integral ideals
    of norm at most $x$ in the
    number field $K$ of degree $n_K$ and discriminant $\Delta_K$ is approximated given by
    $$
    I_K(x)
    =\kappa_K x
    +\mathcal{O}^*\biggl(C(K)|\Delta_K|^{\frac{1}{n_K+1}}(\log|\Delta_K|)^{n_K-1}
    x^{1-\frac{2}{n_K+1}}\biggr)
    $$
    where
    $$
    C(K)=0.17\frac{6n_K-2}{n_K-1}
    2.26^{n_K} e^{4n_K+(26/n_K)}n_K^{n_K+(1/2)}
    \biggl(
    44.39\biggl(\frac{82}{1000}\biggr)^{n_K}n_K!
    +
    \frac{13}{n_K-1}
    \biggr).
    $$
\end{thm}


For $n_K=2$, the constant $\Delta_K$ is about $8.80\cdot 10^{11}$. The
constant arising from Sunley's work was about $1.75\cdot 10^{30}$.
\par 
For $n_K=3$, the constant $C(K)$ is approximately equal to $8.45\cdot 10^{11}$. The
constant arising from Sunley's work was about $8.57\cdot 10^{44}$.

An approximation not relying on the discriminant but on the regulator
and the class number has been given in Corollary 2 of
\cite{Debaene*19}.
\par 
\begin{thm}{Theorem (2019)}

  The notation being as above, we have
  $$
  I_K(x)=\kappa_K x
  +O^*\biggl(n_K^{10n_K^2}
  (\text{Reg}_Kh_K)^{1/n_K}\biggr)
  (1+\log\text{Reg}_Kh_K)^{\frac{(n_K-1)^2}{n_K}}
  x^{1-\frac{1}{n_K}}.
  $$
\end{thm}

The reader will also find there an explicit bound of similar strength
on the number ideals in a given ideal class. 
The number of integral ideals in a given ray class is approximated
explicitely following the same scheme in Theorem 1 of
\cite{Gun-Ramare-Sivaraman*22b}.

\section{Bounds for the residue of the Dedekind zeta-function}

Let $K$ be a number field over $\mathbb{Q}$ with degree $n_K$ and
discriminant $\Delta_K$. Furthermore, suppose that the residue of the
Dedekind zeta function $\zeta_K(s)$ at $s=1$ is denoted
$\kappa_K$. Unconditional bounds for the residue of the Dedekind
zeta-function at $s=1$ are found in
\cite{Louboutin*00} and in
Section 3 of
\cite{Garcia-Lee*22}.
\par 
\begin{thm}{Theorem (2000, 2022)}

If $n_K\geq 3$, then
$$
    \frac{0.0014480}{n_K g(n_K){|\Delta_K|}^{1/n_K}}
    < \kappa_K \leq
    \left(\frac{e\log |\Delta_K| }{2(n_K - 1)}\right)^{n_K - 1},
$$
in which $g(n_K)=1$ if $K$ has a normal tower over $\mathbb{Q}$ and
      $g(n_K) = n_K!$ otherwise.
\end{thm}


If the Generalised Riemann Hypothesis and Dedekind Conjecture
(i.e. $\zeta_K/\zeta$ is entire) are
true, then stronger bounds are found in Corollary 2 of
\cite{Garcia-Lee*22b}.
\par 
\begin{thm}{Theorem (2022)}

    Assume that the Generalised Riemann Hypothesis and the Dedekind
    Conjecture are true. If $n_K\geq 2$, then 
$$
    \frac{e^{-17.81(n_K -1)}}{\log\log{|\Delta_K|}} \leq \kappa_{K}
    \leq e^{17.81(n_K -1 )} (\log\log{|\Delta_K|})^{n_K-1} . 
$$
\end{thm}

    


\section{Zeroes and zero-free regions }


We denote by $N_K(T)$ the number of zeros $\rho$, of the Dedekind
zeta-function of the number field $K$ of degree $n$ and discriminant
$d_K$,
zeros that lie in the critical strip
$0 < \Re \rho = \sigma < 1$ and which verify $|\Im \rho|\le T$.
After a first result in
\cite{Kadiri-Ng*12},
we find in
\cite{Trudgian*14-1}
the following result.

\par 
\begin{thm}{Theorem (2014)}

 When $T\ge1$, we have
 $N_K(T)=\frac{T}{\pi}\log\Bigl(|d_K|\Big(\frac{T}{2\pi e}\Bigr)^n\Bigr)
 +O^*\bigl(0.316(\log |d_K|+n\log T)+5.872 n+3.655\bigr)$ .
\end{thm}


This is improved in 
\cite{Hasanalizade-Shen-Wong*21}
into:

\par 
\begin{thm}{Theorem (2021)}

 When $T\ge1$, we have
 $N_K(T)=\frac{T}{\pi}\log\Bigl(|d_K|\Big(\frac{T}{2\pi e}\Bigr)^n\Bigr)
 +O^*\bigl(0.228(\log |d_K|+n\log T)+23.108 n+4.520\bigr)$ .
\end{thm}


In
\cite{Kadiri*12},
a zero-free region is proved.

\par 
\begin{thm}{Theorem (2012)}

Let $K$ be a number field of degree $n$ over $\mathbb{Q}$ and of
discriminant $d_K$ such that $|d_K| \ge 2$. The associated Dedekind
zeta-function $\zeta_K$ has no zeros in the region
$$
\sigma\ge 1-\frac{1}{12.55\log|d_K|+n(9.69\log|t|+3.03)+58.63}, |t|\ge1
$$
and at most one zero in the region
$$
\sigma\ge 1-\frac{1}{12.74\log|d_K|}, |t|\le 1.
$$
The exceptional zero, if it exists, is simple and real.
\end{thm}


This is improved in
\cite{Lee*21a}
into:
\par 
\begin{thm}{Theorem (2021)}

    Let $K$ be a number field of degree $n$ over $\mathbb{Q}$ and of
discriminant $d_K$ such that $|d_K| \ge 2$. The associated Dedekind
zeta-function $\zeta_K$ has no zeros in the region
$$
\sigma\ge 1-\frac{1}{12.2411\log|d_K|+n(9.5347\log|t|+0.0501)+2.2692}, |t|\ge1 
$$
and if $d_K$ is sufficiently large, then there is at most one zero in the region
$$
\sigma\ge 1-\frac{1}{12.4343\log|d_K|}, |t|< 1.
$$
The exceptional zero, if it exists, is simple and real.
\end{thm}


See
\cite{Ahn-Kwon*14}
for a result for Hecke $L$-series.

In

In
\cite{Louboutin*17},
a zero-free region is proved. Here is slightly simplified version of
his result.

\par 
\begin{thm}{Theorem (2017)}

Let $K$ be a number field of degree $n$ over $\mathbb{Q}$ and of
discriminant $d_K$ such that $|d_K| \ge 8$. The associated Dedekind
zeta-function $\zeta_K$ has no zeros in the regions
$$
\sigma\ge 1-\frac{1}{1.7\log|d_K|}, |t|\ge\frac{1}{4\log|d_K|},
$$
$$
\sigma\ge 1-\frac{1}{2\log|d_K|}, |t|\ge\frac{1}{2\log|d_K|},
$$
and
$$
\sigma\ge 1-\frac{1}{1.62\log|d_K|}, t=0.
$$
\end{thm}



\section{. Real Zeroes.}


In
\cite{Louboutin*15b},
we find the next result.

\par 
\begin{thm}{Theorem (2015)}

Let $m$ be a positive integer.
  Let $K$ be a number field of degree $n$ over $\mathbb{Q}$ and of
discriminant $d_K$ such that $|d_K|\ge
\exp((5+\sqrt{5})(\sqrt{m+1}-1)^2)$.
The associated Dedekind
zeta-function $\zeta_K$ has at most $m$ real zeroes in
$$
\sigma\ge 1-\frac{(5+\sqrt{5})(\sqrt{m+1}-1)^2}{2\log|d_K|}.
$$
\end{thm}







  
\begin{flushright}\small\sl{}   Last updated on January 15th, 2024, by Ethan Lee;
 \end{flushright}














