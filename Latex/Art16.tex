\chapter{  Tools on Fourier transforms}

Corresponding html file: \texttt{../Articles/Art16.html}










\section{The large sieve inequality}


The best version of the large sieve inequality from
\cite{Montgomery-Vaughan*74}
and
\cite{Montgomery-Vaughan*73}
(obtained at the same time by A. Selberg) is as follows.
\par 
\begin{thm}{Theorem (1974)}

Let $M$ and $N\ge 1$ be two real numbers. Let $X$ be a set of points of
$[0,1)$ such that 
$
\displaystyle
\min_{\substack{x,y\in X\\ x\neq y}}
\min_{k\in\mathbb{Z}}|x-y+k|\ge \delta>0.
$
Then, for any sequence of complex numbers $(a_n)_{M < n\le M+N}$, we have
$$
\sum_{x\in X}\left|
\sum_{M < n\le M+N} a_n \exp(2i\pi nx)
\right|^2
\le \sum_{M < n\le M+N}|a_n|^2 (N-1+\delta^{-1}).
$$
\end{thm}


It is very often used with part of the Farey dissection.
\par 
\begin{thm}{Theorem (1974)}

Let $M$ and $N\ge 1$ be two real numbers. Let $Q\ge1$ be a real parameter.
For any sequence of complex numbers $(a_n)_{M < n\le M+N}$, we have
$$
\sum_{q\in Q}\sum_{\substack{a\mod q,\\ (a,q)=1}}\left|
\sum_{M < n\le M+N} a_n \exp(2i\pi na/q)
\right|^2
\le \sum_{M < n\le M+N}|a_n|^2 (N-1+Q^2).
$$
\end{thm}

The summation over $a$ runs over all invertible (also called reduced)
classes $a$ modulo $q$.

The weighted large sieve inequality from Theorem 1 in
\cite{Montgomery-Vaughan*74}
reads as follows.
\par 
\begin{thm}{Theorem (1974)}

Let $M$ and $N\ge 1$ be two real numbers. Let $X$ be a set of points of
$[0,1)$. Define
$
\displaystyle
\delta(x)=\min_{\substack{y\in X\\ y\neq x}}
\min_{k\in\mathbb{Z}}|x-y+k|.
$
Then, for any sequence of complex numbers $(a_n)_{M < n\le M+N}$, we have
$$
\sum_{x\in X}\left|
\bigl(N+c\delta(x)^{-1}\bigr)\sum_{M < n\le M+N} a_n \exp(2i\pi nx)
\right|^2
\le \sum_{M < n\le M+N}|a_n|^2
	      $$
	      for $c=3/2$.
\end{thm}

It is expected that one can reduce the constant $c=3/2$ to 1. In this
direction, we find in
\cite{Preissmann*84}
the next result.
\par 
\begin{thm}{Theorem (1984)}

We may take $\displaystyle c= \sqrt{1+\tfrac23\sqrt{\tfrac65}} =
1.3154\cdots < 4/3$ in the previous theorem.
\end{thm}



 
 








  
\begin{flushright}\small\sl{}   Last updated on February 7th, 2024, by Olivier Ramar\'e
 \end{flushright}















