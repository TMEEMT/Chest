\chapter{   Explicit results on exponential sums}

Corresponding html file: \texttt{../Articles/Art05.html}










Collecting references:
\cite{Daboussi-Rivat*01}.



 
 


\par 
\section{Bounds with the first derivative}


We start with the Kusmin-Landau Lemma.
\par 
\begin{thm}{Theorem}

    Let $f$ be a function over $[a, b]$ such that $f^\prime$ is monotonic
    and satisfies $\theta \le f^\prime(u)\le 1-\theta$ for some $\theta
    \in(0,1/2]$. Then
    $$
    \biggl|\sum_{a\le n\le b} e(f(n))\biggr|
    \le
    \cot\frac{\pi\theta}{2}\le \frac{2}{\pi \theta}.
    $$
\end{thm}



\par 
\section{Bounds with the second derivative}


  Here is a corrected version of Lemma 3 of
  \cite{Cheng-Graham*01}.
\par 
\begin{thm}{Theorem (2004)}

  Let $f$ be a real-valued function with two continuous derivatives
  on $[N+1, N+L]$. Suppose there are $W > 1$ and $\lambda > 1$ such
  that $1 \le W |f^{\prime\prime\prime}(x)| \le \lambda$ for every $x\in [N+1,
  N+L]$. Then we have
  $$
  \biggl|\sum_{n= N+1}^{N+L}
  \exp( 2i\pi f(n)) \biggr|
  \le 2\biggl(\frac{L \lambda}{W} +2\biggr)
  \biggl(2\sqrt{\frac{W}{\pi}} + 1\biggr).
  $$
\end{thm}



  
\par 
\section{Bounds with the third derivative}


Here is Lemma 1.2 of
\cite{Hiary*16}.
\par 
\begin{thm}{Theorem (2016)}

  Let $f$ be a real-valued function with three continuous derivatives
  on $[N+1, N+L]$. Suppose there are $W > 1$ and $\lambda > 1$ such
  that $1 \le W |f^{\prime\prime\prime}(x)| \le \lambda$ for every $x\in [N+1,
  N+L]$. Then, for any $\eta > 0$, we have
  $$
  \biggl|\sum_{n= N+1}^{N+L}
  \exp( 2i\pi f(n)) \biggr|^2
  \le (LW^{-1/3} +\eta) (\alpha L + \beta W^{2/3})
  $$
  where
  $$
  \alpha = \frac{1}{\eta} +\frac{64\lambda}{75}
  \sqrt{\eta + W^{-1/3}}+\frac{\lambda\eta}{W^{1/3}}
  +\frac{\lambda}{W^{2/3}},
  $$
  and
  $$
  \beta = \frac{65}{15\sqrt{\eta}} + \frac{3}{W^{1/3}}.
  $$
\end{thm}


  
\par 
\section{Iterated Van der Corput Inequality}


During the proof of Lemma 8.6 in 
  \cite{Granville-Ramare*96},
  one finds the next inequality.
\par 
\begin{thm}{Theorem (1996)}

  Let $f$ be a real-valued function with $k+1$ continuous derivatives
  on $(A, B]$ and let $N$ be a lower bound for the number of integers
  in $(A,B]$. The quantity
  $$
  \biggl|\frac{1}{8N}
  \sum_{A < n\le B} \exp(2 i \pi f(n))\biggr|^{2^k}
	    $$
	    is bounded above by
	    $$
	    \frac{1}{8}\biggl(
	    \frac{1}{Q} + \frac{1}{Q^{2-2^{1-k}}}
	    \sum_{r_1 =1}^{Q2^{-0}}
	    \sum_{r_2 =1}^{Q2^{-1}}
	    \cdots
	    \sum_{r_k =1}^{Q2^{-k+1}}
	    \biggl|
	    \frac{1}{N}
	    \sum_{A < n \le B-r_1-r_2-\cdots-r_k}
		    \exp(\pm 2i\pi f_{r_1,\cdots,r_k}(n))   
	    \biggr|
	    \biggr)
		      $$
		      where the function $f_{r_1,\cdots,r_k}$ satisfies
		      $$
		      \forall t,\ \exists y\in[t, t + r_1 + \cdots + r_k],
		      \quad
		      f^{\prime}_{r_1,\cdots, r_k}(t) = r_1r_2\cdots r_k f^{(k+1)}(y).
		      $$
\end{thm}



\par 
\section{Explicit Poisson Formula}


Here is a consequence of the main theorem of
\bibref(Karatsuba-Korolev*07)
.
\par 
\begin{thm}{Theorem (2007)}

    Suppose $f^\prime$ is decreasing on $[N+1,N+L]$ and set
    $f^\prime(N+L)=\alpha$ and $f^\prime(N) = \beta$.
    For integer $\nu\in(\alpha, \beta]$, let $x_\nu$ be the solution
    to $f^\prime(x)=\nu$. Suppose further that
$\lambda_2\le |f^{\prime\prime}(x)|\le h_2\lambda_2$ and
    $\lambda_3\le |f^{\prime\prime\prime}(x)|\le h_3\lambda_3$. Then
    $$
    \sum_{n=N+1}^{N+L} \exp(2i\pi f(n))
    =
    \sum_{\alpha < \nu\le \beta}
		   \frac{\exp(2i\pi (f(x_\nu)-\nu x_\nu-1/8))}{\sqrt{f^{\prime\prime}(x_\nu)}}
		   +\mathcal{E}
    $$
		   where
		   $$
		   |\mathcal{E}|
		   \le \frac{40}{\sqrt{\pi}}\lambda^{-1/2}
		   + \frac{3+2h_2}{\pi} \log(\beta-\alpha+2)
		   + 2.9 h_2h_3^{1/5}L(\lambda_2\lambda_3)^{1/5}
		   +1.9.
		   $$
		   
\end{thm}








  
\begin{flushright}\small\sl{}   Last updated on July 12th, 2023, by Olivier Ramar\'e
 \end{flushright}















