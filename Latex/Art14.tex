\chapter{  Sieve bounds}

Corresponding html file: \texttt{../Articles/Art14.html}










 
 

\par 
\section{Some upper bounds}


Theorem 2 of \cite{Montgomery-Vaughan*73} contains the
following explicit version of the Brun-Tichmarsh Theorem.
\par 
\begin{thm}{Theorem (1973)}

Let $x$ and $y$ be positive real numbers, and let $k$ and $\ell$ be relatively
prime positive integers. Then 
$
\pi(x+y;k,\ell)-\pi(x;k,\ell)
<  \frac{2y}{\phi(k)\log (y/k)}
$ provided only that $y>k$.
\end{thm}

Here as usual, we have used the notation
$$
\pi(z;k,\ell)=\sum_{\substack{p\le z,\\ p\equiv \ell [k]}}1,
$$
i.e. the number of primes up to $z$ that are coprime to $\ell$ modulo $k$.
See
\cite{Buethe*14}
for a generic weighted version of this inequality.

Lemma 14 of
\cite{Ramare*13a},
the following extension of the above is proved.
\par 
\begin{thm}{Theorem (2021)}

Let $x\ge y>k\ge 1$ be positive real numbers, $k$ being an integer. Then 
$
\displaystyle
\sum_{\substack{x < m \le x+y\\ m\equiv a[k]}}\frac{\Lambda(m)}{\log m}
<  \frac{2y}{\phi(k)\log (y/k)}.
$
\end{thm}

And in Lemma 15 of the same paper, we find the next estimate.
\par 
\begin{thm}{Theorem (2021)}

Let $x\ge \max(121,k^3)$. Then 
$
\displaystyle
\sum_{\substack{x < m \le 2x\\ m\equiv a[k]}}\Lambda(m)
<  \frac{9}{2}\frac{x}{\phi(k)}.
$
\end{thm}


\par 
Here is a bound concerning a sieve of dimension 2 proved by
\cite{Siebert*76}.
\par 
\begin{thm}{Theorem (1976)}

Let $a$ and $b$ be coprime integers with $2|ab$. Then we have, for $x>1$,
$$
\sum_{\substack{p\le x,\\ \text{$ap+b$ prime}}}1
\le 16 \omega\frac{x}{(\log x)^2}\prod_{\substack{p|ab,\\ p >
2}}\frac{p-1}{p-2}
\qquad \omega=\prod_{p > 2}(1-(p-1)^{-2}).
$$
\end{thm}


This is improved for large values in Lemma 4 of
\cite{Riesel-Vaughan*83}.
\par 
\begin{thm}{Theorem (1983)}

Let $a$ and $b$ be coprime integers with $2|ab$. Then we have, for $x
\ge e^L$,
$$
\sum_{\substack{p\le x,\\ \text{$ap+b$ prime}}}1
\le \biggl(\frac{16 \omega\, x}{(\log x)(A+\log x)}-100\sqrt{x}\biggr)\prod_{\substack{p|ab,\\ p >
2}}\frac{p-1}{p-2}
\qquad \omega=\prod_{p > 2}(1-(p-1)^{-2})
$$
and where

  
  
    
      $L$:
      24
      25
      26
      27
      28
      29
      31
      34
      42
      60
      690
    
    
      $A$:
      0
      1
      2
      3
      4
      5
      6
      7
      8
      8.3
      8.45
    
  
  
  
\end{thm}




\section{Density estimates }


In Theorem 1, page 52 of
\cite{Greaves*01},
we find the next widely applicable estimate.
\par 
\begin{thm}{Theorem (2022)}

    Let $\kappa$ be a non-negative function on the primes such that
    $\kappa(p) < p$. Assume there is a constant $B$ such that
		 $$
		 \sum_{p < z} \frac{\kappa(p)\log p}{p}
			   \le B\log z
			   $$
			   for some $z\ge 2$. Then, when $s\ge 2B$, we
			   have
			   $$
			   \sum_{d\le z^{s/2}}\mu^2(d)
			   \prod_{p|d}\frac{\kappa(p)}{p-\kappa(p)}
			   \ge
			   \Bigl(1-\exp-\bigl(\frac{s}{2}\log\frac{s}{2B}-\frac{s}{2}+B\bigr)\Bigr)
			   \prod_{p < z}\biggl(1-\frac{\kappa(p)}{p}\biggr)^{-1}.
			   $$
\end{thm}

See also here\footnote{\url{Articles/Art10.html#asy}}.


\section{Combinatorial sieve estimates}


The combinatorial sieve is known to be difficult from an explicit
viewpoint. For the linear sieve, the reader may look at Chapter 9,
Theorem 9.7 and 9.8 from
\cite{Nathanson*96-2}.


\section{Integers free of small prime factors}


In
\cite{Fan*22}, 
the following neat estimate is proved.
\par 
\begin{thm}{Theorem (2022)}

Let $\Phi(x,z)$ be the number of integers $\le x$ that do not have any
prime factors $\le z$. We have
$$
\Phi(x,z)\le \frac{x}{\log z},
\quad(1 < z\le x).
$$
\end{thm}







  
\begin{flushright}\small\sl{}   Last updated on February 6th, 2024, by Olivier Ramar\'e
 \end{flushright}














