\chapter{   Bounds for $|\zeta(s)|$, $|L(s,\chi)|$ and related questions}

Corresponding html file: \texttt{../Articles/Art06.html}









Collecting references:
\cite{Trudgian*11},
\cite{Kadiri-Ng*12},

 
 

\par 
\section{Approximating $|\zeta(s)|$ or $L$-series in the}



In
\cite{Reyna*11}
extends the
Phd memoir
\cite{Gabcke*79}
and provides an explicit Riemann-Siegel formula for $\zeta(s)$.

Theorem 1.2 of
\cite{Kadiri*13}
proves the following.
\par 
\begin{thm}{Theorem (2013)}

  When $t > t_0 > 0$, $c > 1/(2\pi)$ and $s = \sigma +it$ with
  $\sigma\ge 1/2$, we have
  $$
  \zeta(s)
  =\sum_{n < c t} \frac{1}{n^s}
	     + \mathcal{O}^* \biggl(
	     \biggl(c+\tfrac12+\frac{3\sqrt{1+1/t_0^2}}{2\pi}
	     \biggl(\frac{\pi}{12c}+1+\frac{1}{2\pi c-1}\biggr)
	     \biggr)
	     (ct)^{-\sigma}\biggr).
  $$
\end{thm}

Notice that, by using the constant $c$, we may deduce from this an
approximation of $\zeta(s)$ by a fixed Dirichlet polynomial when $T
\le t\le 2T$, for some parameter $T$.


\par 

\section{Size of $|\zeta(s)|$ and of $L$-series}



Theorem 4 of \cite{Rademacher*59} gives
the convexity bound. See also section 4.1 of \cite{Trudgian*13}.
\par 
\begin{thm}{Theorem (1959)}

In the strip $-\eta\le \sigma\le 1+\eta$, $0 < \eta\le 1/2$, the Dedekind zeta
function $\zeta_K(s)$ belonging to the algebraic number field $K$ of degree
$n$ and discriminant $d$ satisfies the inequality
$$
|\zeta_K(s)|\le 3 \left|\frac{1+s}{1-s}\right|
\left(\frac{|d||1+s|}{2\pi}\right)^{\frac{1+\eta-\sigma}{2}}
\zeta(1+\eta)^n.
$$
\end{thm}


On the line $\Re s=1/2$, Lemma 2 of
\cite{Lehman*70} gives a better
result, namely
\par 
\begin{thm}{Theorem (1970)}

If $t\ge 1/5$, we have
$
|\zeta(\tfrac12+it)|\le 4 (t/(2\pi))^{1/4}
$.
\end{thm}

In fact, Lehman states this Theorem for $t\ge 64/(2\pi)$, but modern means of
computations makes it easy to check that it holds as soon as $t\ge 0.2$.
See also equation (56)
of \cite{Backlund*18} reproduced below.

For Dirichlet $L$-series, Theorem 3
of \cite{Rademacher*59} gives 
the corresponding convexity bound.
\par 
\begin{thm}{Theorem (1959)}

In the strip $-\eta\le \sigma\le 1+\eta$, $0 < \eta\le 1/2$, for all moduli $q
> 1$ and all primitive
characters $\chi$ modulo $q$, the inequality 
$$
|L(s,\chi)|\le  
\left(q\frac{|1+s|}{2\pi}\right)^{\frac{1+\eta-\sigma}{2}}
\zeta(1+\eta)
$$
holds.
\end{thm}

This paper contains other similar convexity bounds.


Corollary to Theorem 3
of \cite{Cheng-Graham*01} goes beyond convexity. 

\par 
\begin{thm}{Theorem (2001)}

  For $0\le t\le e$, we have $|\zeta(\tfrac12+it)|\le 2.657$. For $t\ge e$, we
  have $|\zeta(\tfrac12+it)|\le 3t^{1/6}\log t$.
  Section 5 of \cite{Trudgian*13}
  shows that one can replace the constant 3 by 2.38.
\end{thm}


This is improved in
\cite{Hiary*16}.
\par 
\begin{thm}{Theorem (2016)}

  When $t\ge 3$, we
  have $|\zeta(\tfrac12+it)|\le 0.63t^{1/6}\log t$.
\end{thm}


Concerning $L$-series, the situation is more difficult but
\cite{Hiary*16b}
manages, among other and more precise results, to prove the following.
\par 
\begin{thm}{Theorem (2016)}

  Assume $\chi$ is a primitive Dirichlet character modulo $q>1$. Assume
  further that $q$ is a sixth power. Then, when $|t|\ge 200$, we
    have
    $$|L(\tfrac12+it,\chi)|\le 9.05d(q) (q|t|)^{1/6}(\log
    q|t|)^{3/2}$$
    where $d(q)$ is the number of divisors of $q$.
\end{thm}


It is often useful to have a representation of the Riemann zeta function
or of $L$-series inside the critical strip. In the case of $L$-series,
\cite{Spira*69}
and
\cite{Rumely*93}
proceed via decomposition in Hurwitz zeta function which they compute through
an Euler-MacLaurin development. We have a more efficient approximation of the
Riemann zeta function provided by the Riemann Siegel formula, see
for instance equations (3-2)--(3.3)
of \cite{Odlyzko*87}. This
expression is due to 
\cite{Gabcke*79}.
See also 
equations (2.4)-(2.5) of
\cite{Lehman*66}, a corrected
version of Theorem 2 of \cite{Titchmarsh*47}.

\par 
In general, we have the following estimate taken from equations
(53)-(54), (56) and (76)
of  \cite{Backlund*18}
(see also \cite{Backlund*14}).
\par 
\begin{thm}{Theorem (1918)}

  \begin{itemize}
  \item When $t\ge 50$ and $\sigma\ge1$, we have $|\zeta(\sigma+it)|\le \log
  t-0.048$.

  \item  When $t\ge 50$ and $0\le \sigma\le1$, we have $|\zeta(\sigma+it)|\le
  \frac{t^2}{t^2-4}\left(\frac{t}{2\pi}\right)^{\frac{1-\sigma}{2}}\log t$.
  

  \item  When $t\ge 50$ and $-1/2\le \sigma\le0$, we have $|\zeta(\sigma+it)|\le
  \left(\frac{t}{2\pi}\right)^{\frac{1}{2}-\sigma}\log t$.
  

  \end{itemize}
\end{thm}



On the line $\Re s=1$, one can rely on
\cite{Trudgian*12b}.
\par 
\begin{thm}{Theorem (2012)}

  When $t\ge 3$, we have $|\zeta(1+it)|\le\tfrac34 \log t$.
\end{thm}


Asymptotically better bounds are available since the huge work of
\cite{Ford*02}.
\par 
\begin{thm}{Theorem (2002)}

  When $t\ge 3$ and $1/2\le \sigma\le 1$, we have $|\zeta(\sigma+it)|\le 76.2
  t^{4.45(1-\sigma)^{3/2} } (\log t)^{2/3}$.
\end{thm}

The constants are still too large for this result to be of use in any decent
region. See \cite{Kulas*94} for an
earlier estimate.





\par 
\section{On the total number of zeroes}


The first explicit estimate for the number of zeros of the Riemann
$\zeta$-function goes back to
\cite{Backlund*14}.
An elegant consequence of the result of Backlund is the following easy
estimate taken from Lemma 1 of
\cite{Lehman*66a}.

\begin{thm}{Theorem (1966)}

If $\varphi$ is a continuous function which is positive and monotone
decreasing for $2\pi e\le T_1\le t\le T_2$, then
$$
\sum_{T_1 < \gamma\le T_2} \varphi(\gamma)
            =\frac{1}{2\pi}\int_{T_1}^{T_2}\varphi(t)\log\frac{t}{2\pi}dt
            +O^*\biggl(4\varphi(T_1)\log
            T_1+2\int_{T_1}^{T_2}\frac{\varphi(t)}{t}
            dt\biggr)
            $$
             where the summation is over all zeros of the Riemann
            $\zeta$-function of
            imaginary part between $T_1$ and $T_2$, with multiplicity.
\end{thm}


Theorem 19 of
\cite{Rosser*41}
gives a bound for the total number of zeroes.

\begin{thm}{Theorem (1941)}

For $T\ge2$, we have
$$
N(T)=\sum_{\substack{\rho,\\ 0 < \gamma\le T}} 1=
            \frac{T}{2\pi}\log\frac{T}{2\pi}-\frac{T}{2\pi}+\frac{7}{8}
            +O^*\Bigl(0.137\log T+0.443\log\log T+1.588
            \Bigr)
            $$
            where the summation is over all zeros of the Riemann
            $\zeta$-function of
            imaginary part between 0 and $T$, with multiplicity.
\end{thm}


It is noted in Lemma 1 of
\cite{Ramare-Saouter*02}
that the $O$-term can be replaced by the simpler
$O^*(0.67\log\frac{T}{2\pi})$ when $T\ge 10^3$.

This is improved in Corollary 1 of
\cite{Trudgian*13}
into
\begin{thm}{Theorem (2014)}

For $T\ge e$, we have
$$
N(T)=\sum_{\substack{\rho,\\ 0 < \gamma\le T}} 1=
            \frac{T}{2\pi}\log\frac{T}{2\pi}-\frac{T}{2\pi}+\frac{7}{8}
            +O^*\bigl(0.112\log T+0.278\log\log T+2.510+\frac{1}{5T}
            \bigr)
            $$
            where the summation is over all zeros of the Riemann
            $\zeta$-function of
            imaginary part between 0 and $T$, with multiplicity.
\end{thm}



Corollary 1.4 of the main theorem of
\cite{Hasanalizade-Shen-Wong*22}
reads
\begin{thm}{Theorem (2022)}

For $T\ge e$, we have
$$
N(T)=\sum_{\substack{\rho,\\ 0 < \gamma\le T}} 1=
            \frac{T}{2\pi}\log\frac{T}{2\pi}-\frac{T}{2\pi}+\frac{7}{8}
            +O^*\bigl(0.1038\log T+0.2573\log\log T+9.3675
            \bigr)
            $$
            where the summation is over all zeros of the Riemann
            $\zeta$-function of
				 imaginary part between 0 and $T$, with multiplicity.
				 We may also replace $0.1038\log
				 T+0.2573\log\log T+9.3675$ by $0.1095\log T+0.2042\log\log T+3.0305$.
\end{thm}



Concerning Dirichlet $L$-functions, the paper
\cite{Bennett-Martin-OBryant-Rechnitzer*21}
contains the next result.

\begin{thm}{Theorem (2021)}

    Let $\chi$ be a Dirichlet character of conductor $q > 1$.
For $T\ge 5/7$ and $\ell= \log\frac{q(T+2)}{2\pi} > 1.567$, we have
$$
N(T,\chi)=\sum_{\substack{\rho,\\ 0 < \gamma\le T}} 1=
            \frac{T}{\pi}\log\frac{qT}{2\pi}-\frac{T}{\pi}+\frac{\chi(-1)}{4}
            +O^*\bigl(0.22737\ell+2\log(1+\ell)-0.5
            \bigr)
            $$
            where the summation is over all zeros of the Dirichlet
            function $L(\cdot,\chi)$ of
				 imaginary part between $-T$ and $T$, with multiplicity.
				 
\end{thm}





\par 
\section{L${}^2$-averages}


In Theorem 1.4 of
\cite{Kadiri*13},
we find the next result.
\begin{thm}{Theorem (2013)}

    When $0.5208 < \sigma < 0.9723$ and $10^3\le H \le 10^{10}$,
    we have, for any $T\ge H$,
    $$
			    \int_H^T |\zeta(\sigma + t)|^2 dt
			    \le
			    (T-H) \bigl(\zeta(2\sigma) +
    \mathcal{E}_1(\sigma, H)\bigr)
			    $$
    where $\mathcal{E}_1(\sigma, H)$ is a small error term whose
    precise expression in given the stated paper.
\end{thm}


We can find in
\cite{Helgott*17u} the proof
of the following estimate. Though it is unpublished yet, the full proof
is available.
\begin{thm}{Theorem (2019)}

Let $0 < \sigma\le1$ and $T \ge 3$. Then
  $$
    \frac{1}{2\pi}\biggl(
    \int_{\sigma-i\infty}^{\sigma-iT}
    +
    \int^{\sigma+i\infty}_{\sigma+iT}
    \biggr)
    \frac{|\zeta(s)|^2}{|s|^2}ds\le
    \kappa_{\sigma,T}
    \begin{cases}
    \frac{c_{1,\sigma}}{T}+\frac{c^\flat_{1,\sigma}}{T^{2\sigma}}
    &\text{when $\sigma > 1/2$,}\\
    \frac{\log T}{2T}+\frac{c^\flat_{2,\sigma}}{T}
    &\text{when $\sigma=1/2$,}\\
    c_{3,\sigma}/T^{2\sigma}&\text{when $\sigma < 1/2$.}
    \end{cases}
  $$
 where
$$
c_{1,\sigma}=\zeta(2\sigma)/2,
 c_{1,\sigma}^\flat=c^2 \frac{3^{2\sigma}}{2\sigma},
  c_{2,\sigma}^\flat=3c^2+\frac{1-\log 3}{2},
c=9/16,						  
 $$
 $$
 c_{3,\sigma}=
\Bigl(\frac{c^2}{2\sigma}+\frac{1/6}{1-2\sigma}\Bigr)
\Bigl(1+\frac{1}{\sigma}\Bigr)^{2\sigma},
\kappa_{\sigma,T}=
\begin{cases}
\frac{9/4}{\left(1-\frac{9/2}{T^2}\right)^2}
&\text{when $1/2\le \sigma\le 1$,}\\
\frac{(1+\sigma)^2}{\left(1-\frac{(1+\sigma)^2}{\sigma T^2}\right)^2}
&\text{when $0 < \sigma < 1/2$.}
 \end{cases}
 $$
						  
\end{thm}





\par 
\section{Bounds on the real line}


After some estimates
from \cite{Bastion-Rogalaski*02}, 
Lemma 5.1 of \cite{Ramare*13d} shows
the following.
\par 
\begin{thm}{Theorem (2013)}

  When $\sigma> 1$ and $t$ is any real number, we have $|\zeta(\sigma+it)|\le   e^{\gamma(\sigma-1) }/(\sigma-1)$.
\end{thm}


Here is the Theorem of
\cite{Delange*87}.
See also Lemma 2.3 of
\cite{Ford*01} for a
slightly weaker version.
\par 
\begin{thm}{Theorem (1987)}

  When $\sigma> 1$ and $t$ is any real number, we have
  $$
  -\Re\frac{\zeta'}{\zeta}(\sigma+it)\le
  \frac{1}{\sigma-1}-\frac{1}{2\sigma^2}.
  $$ 
\end{thm}








  
\begin{flushright}\small\sl{}   Last updated on July 12th, 2023, by Olivier Ramar\'e
 \end{flushright}















