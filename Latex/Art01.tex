\chapter{   Explicit bounds on primes}

Corresponding html file: \texttt{../Articles/Art01.html}










 
 


Collecting references:
\cite{Dusart*98},
\cite{Dusart*07},

\section{Bounds on primes, in special ranges}

The paper
\cite{Rosser-Schoenfeld*62},
contains several bounds valid only when the variable is small enough.

In 
\cite{Buthe*16},  the
author proves the next theorem.
\par 
\begin{thm}{Theorem (2016)}

    Assume the Riemann Hypothesis has been checked up to height
    $H_0$. Then when $x$ satisfies $\sqrt{x/\log x}\le H_0/4.92$, we have
  \begin{itemize}
    \item $|\psi(x)-x|\le \frac{\sqrt{x}}{8\pi}\log^2x$ when $x > 59$,

    \item $|\theta(x)-x|\le \frac{\sqrt{x}}{8\pi}\log^2x$ when $x > 599$,

    \item $|\pi(x)-\text{li}(x)|\le \frac{\sqrt{x}}{8\pi}\log x$ when
    $x > 2657$.

					\end{itemize}
\end{thm}

If we use the value $H_0=30\,610\,046\,000$ obtained by D. Platt
in
\cite{Platt*17}, these
bounds are thus valid for $x\le 1.8\cdot 10^{21}$.

In
\cite{Buthe*18},
the following bounds are also obtained.
\par 
\begin{thm}{Theorem (2018)}

    We have
  \begin{itemize}
    \item $|\psi(x)-x|\le 0.94\sqrt{x}$ when $11 < x\le 10^{19}$,

   \item $0<\text{li}(x)-\pi(x)\le\frac{\sqrt{x}}{\log
   x}\Bigl(1.95+\frac{3.9}{\log x}+\frac{19.5}{\log^2x}\Bigr)$  when
   $2\le x\le 10^{19}$.

\end{itemize}
\end{thm}



\section{Bounds on primes, without any congruence condition}


The subject really started with the four papers
\cite{Rosser*41},
\cite{Rosser-Schoenfeld*62},
\cite{Rosser-Schoenfeld*75}
and
\cite{Schoenfeld*76}.
We recall the usual notation: $\pi(x)$ is the number of primes up to
$x$ (so that $\pi(3)=2$), the function $\psi(x)$ is the summatory
function of the van Mangold function $\Lambda$,
i.e. $\psi(x)=\sum_{n\le x}\Lambda(n)$, while we also define
$\vartheta(x)=\sum_{p\le x}\log p$.
Here are some elegant bounds that one can find in these papers.
\par 
\begin{thm}{Theorem (1962)}

  \begin{itemize}
    \item For $x > 0$, we have
      $\psi(x)\le 1.03883 x$ and the maximum of $\psi(x)/x$ is
      attained at $x=113$.

    \item When $x\ge17$, we have $\pi(x) > x/\log x$.

    \item When $x > 1$, we have $\displaystyle \sum_{p\le x}1/p >  \log\log x$.

    \item When $x > 1$, we have $\displaystyle \sum_{p\le x}(\log p)/p <
									\log x$.

									\end{itemize}
\end{thm}

There are many other results in these papers.
In
\cite{Dusart*99-1},
on can find among other things the inequality
$$
\text{When $x\ge17$, we have } \pi(x) > \frac{x}{\log x -1}.
        $$
        And also
\par 
\begin{thm}{Theorem (1999)}

  \begin{itemize}
    \item When $x\ge e^{22}$, we have
  $\displaystyle\psi(x)=x+O^*\Bigl(0.006409\frac{x}{\log
      x}\Bigr)$.

    \item  When $x\ge 10\,544\,111$, we have $\displaystyle\vartheta(x)=x+O^*\Bigl(0.006788\frac{x}{\log
      x}\Bigr)$.

    \item  When $x\ge 3\,594\,641$, we have $\displaystyle\vartheta(x)=x+O^*\Bigl(0.2\frac{x}{\log^2
      x}\Bigr)$.

    \item  When $x > 1$, we have $\displaystyle\vartheta(x)=x+O^*\Bigl(515\frac{x}{\log^3
      x}\Bigr)$.

    \end{itemize}
\end{thm}

This is improved in
\cite{Dusart*16},
and in particular, it is shown that the 515 above can be replaced by
20.83 and also that
$$
\text{When $x\ge 89\,967\,803$, we have } \vartheta(x)=x+O^*\Bigl(\frac{x}{\log^3
x}\Bigr).
$$

Bounds of the shape $|\psi(x)-x|\le \epsilon x$ have started appearing
in
\cite{Rosser-Schoenfeld*62}.
The latest paper is
\cite{Kadiri-Faber*13}
with its corrigendum
\cite{Kadiri-Faber*18},
where the explicit density estimate from
\cite{Kadiri*13}
is put to contribution, even for moderate
values of the variable. In particular
$$
\text{When $x\ge 485\,165\,196$, we have } \psi(x)=x+O^*(0.00053699\,x).
$$

In
\cite{Platt-Trudgian*21b},
one find the following estimate
\par 
\begin{thm}{Theorem (2021)}

  When $x\ge e^{2000}$, we have
  $\biggl|\frac{\psi(x)-x}{x}\biggr|\le 235\,(\log
  x)^{0.52}\exp-\sqrt{\frac{\log x}{5.573412}}\;.$



Refined bounds for $\pi(x)$ are to be found in
\cite{Panaitopol*00}
and in
\cite{Axler*16}.



By comparing in an efficient manner with $\psi(x)-x$,
\cite{Ramare*12-1},
obtained the next two results. [There was an error in this paper which
is corrected below].

\par 
Theorem (2013)


  For $x > 1$, we have
  $\sum_{n\le x}\Lambda(n)/n=\log x-\gamma+O^*(1.833/\log^2x)$.
  When $x\ge 1\,520\,000$, we can replace the error term by $O^*(0.0067/\log
  x)$. When $x\ge 468\,000$, we can replace the error term by $O^*(0.01/\log
  x)$. 
  Furthermore, when $1\le x\le 10^{10}$, this error term can be
  replaced by $O^*(1.31/\sqrt{x})$. 
\end{thm}


\par 
\begin{thm}{Theorem (2013)}

  For $x\ge 8950$, we have
  $$
  \sum_{n\le x}\Lambda(n)/n=\log x-\gamma
  +\frac{\psi(x)-x}{x}+O^*\Bigl(\frac{1}{2\sqrt{x}}+1.75\cdot 10^{-12}\Bigl)
  $$.
\end{thm}



\cite{Vanlalnagaia*15-1}
developed the method to incorporate more functions (and corrected the
initial work), extending results of
\cite{Rosser-Schoenfeld*62}.

Here are some of his results.

\par 
\begin{thm}{Theorem (2017)}

  For $x\ge 2$, we have
  $$
  \sum_{p\le x}\frac1p=\log\log  x+B+O^*\Bigl(\frac{4}{\log^3x}\Bigr).
  $$
  When $x\ge 1000$, one can replace the 4 in the error term by 2.3,
  and when $x\ge24284$, by 1. The constant $B$ is the expected one.
\end{thm}



\par 
\begin{thm}{Theorem (2017)}

  For $\log x\ge 4635$, we have
  $$
  \sum_{p\le x}\frac1p=\log\log
  x+B+O^*\Bigl(1.1\frac{\exp(-\sqrt{0.175\log x})}{(\log x)^{3/4}}\Bigr).
  $$
\end{thm}






\par 
When truncating sums over primes, Lemma 3.2 of
\cite{Ramare*13d}
is handy.
\par 
\begin{thm}{Theorem (2016)}

  Let $f$ be a C${}^1$ non-negative, non-increasing function over
  $[P,\infty[$, where $P\ge 3\,600\,000$ is a real number and such
  that $\lim_{t\rightarrow\infty}tf(t)=0$. 
  We have
  \begin{equation*}
    \sum_{p\ge P} f(p)\log p
    \le (1+\epsilon) \int_P^\infty f(t) dt  +  \epsilon P f(P)  +  P
  f(P) / (5 \log^2 P) 
  \end{equation*}
  with $\epsilon=1/914$. When we can only ensure $P\ge2$, then a similar
  inequality holds, simply replacing the last $1/5$ by a 4.
\end{thm}


The above result  relies on (5.1*) of
\cite{Schoenfeld*76}
because it is easily accessible. However on using
Proposition 5.1 of
\cite{Dusart*07},
one has access to $\epsilon=1/36260$.

\par 
  \par 
Here is a result due to 
\cite{Trevino*12}.
\par 
\begin{thm}{Theorem (2012)}

For $x$ a positive real number. If $x \geq x_0$ then there exist $c_1$
and $c_2$ depending on $x_0$ such that
$$
\frac{x^2}{2\log{x}} +
\frac{c_1 x^2}{\log^2{x}} \leq \sum_{p \leq x} p \leq
\frac{x^2}{2\log{x}} + \frac{c_2 x^2}{\log^2{x}}.
$$
The constants
$c_1$ and $c_2$ are given for various values of $x_0$ in the next
table.

  
  
    
      $x_0$
      $c_1$
      $c_2$
    
  
  
    315437
    0.205448
    0.330479
  
  
    468577
    0.211359
    0.32593
  
  
    486377
    0.212904
    0.325537
  
  
    644123
    0.21429
    0.322609
  
  
    678407
    0.214931
    0.322326
  
  
    758231
    0.215541
    0.321504
  
  
    758711
    0.215939
    0.321489
  
  
    10544111
    0.239818
    0.29251
  
  

\end{thm}

\par 

Further estimates can be found in
 \cite{Axler*19}
(Proposition 2.7 and Corollary 2.8).

  
In
\cite{Deleglise-Nicolas*19a}
(Proposition 2.7 and Corollary 2.8) and
\cite{Deleglise-Nicolas*19b}
(Proposition 2.5, Corollary 2.6, 2.7 and 2.8),
we find among other results the next two.
\par 
\begin{thm}{Theorem (2019)}

  On setting $\pi_r(x)=\sum_{p\le x}p^r$, we have
  $$
  \frac{3x^2}{20(\log x)^4}
  \le \pi_1(x)-\biggl(
  \frac{x^2}{2\log x}
  +\frac{x^2}{4(\log x)^2}
  \frac{x^2}{4(\log x)^3}
  \biggr)
  \le
  \frac{107x^2}{160(\log x)^4}
  $$
  where the upper estimate is valid when $x\ge 110117910$
  and the lower one when $x\ge905238547$.
  \par 
    We have
    $$
  \frac{-1069x^3}{648(\log x)^4}
  \le \pi_2(x)-\biggl(
  \frac{x^3}{3\log x}
  +\frac{x^3}{9(\log x)^2}
  \frac{x^3}{27(\log x)^3}
  \biggr)
  \le
  \frac{11181x^3}{648(\log x)^4}
    $$
    where the upper estimate is valid when $x\ge 60173$
    and the lower one when $x\ge 1091239$.
    \par 
      $$      \pi_3(x)\le 0.271\frac{x^4}{\log x}\quad\text{for $x\ge 664$},$$
      $$      \pi_4(x)\le 0.237\frac{x^5}{\log x}\quad\text{for $x\ge 200$},$$
      $$      \pi_5(x)\le 0.226\frac{x^5}{\log x}\quad\text{for $x\ge 44$},$$
      For $x\ge 1$ and $r\ge5$, we have
      $$ \pi_r(x)\le \frac{\log 3}{3}\bigl(1+(2/3)^r\bigr)
      \frac{x^{r+1}}{\log x}$$.
\end{thm}





  

\section{Bounds containing $n$-th prime}



Denote by $p_n$  the $n$-th prime. Thus $p_1=2,\;p_2=3,\; p_4=5,\cdots$.

The classical form of prime number theorem yields easily
$p_n \sim n \log n.$ 

\cite{Rosser*38}
shows that this equivalence does not oscillate
by proving that $p_n$ is greater than $n\log n$ for $n\geq 2$.  


The asymptotic formula for $p_n$ can be developped as shown in
\cite{Cipolla*02}:
$$
p_n\sim n\left(\log n+\log\log n -1+\frac{\log\log n-2}{\log n}
-\frac{(\ln\ln n)^2-6\log\log n +11}{2\log^2 n}+\cdots\right).
$$

This asymptotic expansion is the inverse of the logarithmic integral
$\mbox{Li}(x)$ obtained by series reversion. 


 But
\cite{Rosser*38}
also proved  that for every $n> 1$:
$$
n (\log n + \log \log n - 10) < p_n < n (\log n + \log\log n +8).
$$
He improves these results in
\cite{Rosser*41}
:  for every $n\geq 55$,
$$
n (\log n + \log \log n - 4) < p_n < n (\log n + \log\log n +2).
$$

 This result was subsequently improved by Rosser and Schoenfeld
\cite{Rosser-Schoenfeld*62}
 in 1962 to
$$
n (\log n + \log \log n - 3/2) < p_n < n (\log n + \log\log n -1/2),
$$
for $n > 1$ and $n > 19$  respectively.

The constants  were subsequently reduced by
\cite{Robin*83-1}.
In particular, the lower bound 
$$
n (\log n + \log \log n - 1.0072629) < p_n
$$
is valid for $n>1$ and the constant $1.0072629$ can be replaced by 1 for
$p_k\leq 10^{11}$.
Then 
\cite{Massias-Robin*96}
  showed that the lower bound constant equals to 1 was admissible for
$p_k < e^{598}$
and $p_k > e^{1800}$. Finally,  
\cite{Dusart*99-2}
showed
that
$$
  n(\log n - \log \log n - 1) < p_n
$$ for all $n > 1$, and also that
$$
p_n < n (\log n + \log\log n - 0.9484)
$$ for $n > 39017$ i.e. $p_n > 467\,473$.

In
\cite{Carneiro-Milinovich-Soundararajan*19},
the authors  prove the next result.
\begin{thm}{Theorem (2019)}

Under the Riemann Hypothesis we have $p_{n+1}-p_n\le\frac{22}{25}\sqrt{p_n}\log p_n$.
\end{thm}


\par 
\par 
In 
\cite{Axler*19},
we find (Theorem 1.6 and 1.7) the next result.
\par 
\begin{thm}{Theorem (2019)}

  We have
  $$
  \sum_{i\le k}p_i\ge
  \frac{k^2}4 +\frac{k^2}{4\log k}
  -\frac{k^2(\log\log k-2.9)}{4(\log k)^2}\quad(\text{for $k\ge 6309751$}),
  $$
  as well as
  $$
  \sum_{i\le k}p_i\le
  \frac{k^2}4 +\frac{k^2}{4\log k}
  -\frac{k^2(\log\log k-4.42)}{4(\log k)^2}\quad(\text{for $k\ge 256376$}),
  $$
\end{thm}


\par 
\par 
In 
\cite{DeKoninck-Letendre*20},
we find in passing (Lemma 4.8) the next result.
\par 
\begin{thm}{Theorem (2020)}

  We have
  $$
  \sum_{i\le k}\log p_i\le
  k\bigl(\log k+\log\log -3/4\Bigr)\quad(\text{for $k\ge 4$}),
  $$
  as well as
  $$
  \sum_{i\le k}\log\log p_i\ge
  k\biggl(\log\log k+\frac{\log\log -5/4}{\log k}\biggr)
  \quad(\text{for $k\ge319$}).
  $$
\end{thm}









\section{Bounds on primes in arithmetic progressions}


Collecting references:
\cite{McCurley*84-2},
\cite{McCurley*84-3},
\cite{Ramare-Rumely*96},
\cite{Dusart*01},
Lemma 10 of \cite{Moree*04},
section 4 of
\cite{Moree-teRiele*04}.

A consequence of Theorem 1.1 and 1.2 of
\cite{Bennett-Martin-OBryant-Rechnitzer*18}
states that
\begin{thm}{Theorem (2018)}

  We have
  $\displaystyle
  \max_{3\le q\le 10^4}\max_{ x\ge 8\cdot 10^9}\max_{\substack{1\le a\le q,\\
  (a,q)=1}}
  \frac{\log x}{x}\Bigl|
  \sum_{\substack{n\le x,\\ n\equiv a[q]}}\Lambda(n)
  -\frac{x}{\varphi(q)}\Bigr|\le 1/840.
  $
  \par 
  When $q\in(10^4, 10^5]$, we may replace $1/840$ by $1/160$ and when
  $q\ge 10^5$, we may replace $1/840$ by $\exp(0.03\sqrt{q}\log^3q)$.
  \par 
    Furthermore, we may replace
  $\sum_{\substack{n\le x,\\ n\equiv a[q]}}\Lambda(n)$ by
  $\sum_{\substack{p\le x,\\ p\equiv a[q]}}\log p$ with no changes in
  the constants. 
\end{thm}

Similarly, as a consequence of Theorem 1.3 of
\cite{Bennett-Martin-OBryant-Rechnitzer*18}
states that
\begin{thm}{Theorem (2018)}

  We have
  $\displaystyle
  \max_{3\le q\le 10^4}\max_{ x\ge 8\cdot 10^9}\max_{\substack{1\le a\le q,\\
  (a,q)=1}}
  \frac{\log^2 x}{x}\Bigl|
  \sum_{\substack{p\le x,\\ p\equiv a[q]}}1
  -\frac{\textrm{Li}(x)}{\varphi(q)}\Bigr|\le 1/840.
  $
  \par 
  When $q\in(10^4, 10^5]$, we may replace $1/840$ by $1/160$ and when
  $q\ge 10^5$, we may replace $1/840$ by $\exp(0.03\sqrt{q}\log^3q)$.
\end{thm}





\par 
Concerning estimates with a logarithmic density, in
\cite{Ramare*02}
and in
\cite{Ramare*12-0},
estimates for the functions
$\displaystyle\sum_{\substack{n\le x,\\ n\equiv a[q]}}\Lambda(n)/n$
are considered.
Extending computations from the former, the latter paper Theorem 8.1
reads as follows.
\begin{thm}{Theorem (2016)}

  We have
  $\displaystyle
  \max_{q\le 1000}\max_{q\le x\le 10^5}\max_{\substack{1\le a\le q,\\
  (a,q)=1}}
  \sqrt{x}\Bigl|
  \sum_{\substack{n\le x,\\ n\equiv a[q]}}\frac{\Lambda(n)}{n}
  -\frac{\log x}{\varphi(q)}-C(q,a)\Bigr|\in(0.8533,0.8534)
  $
  and the maximum is attained with $q=17$, $x=606$ and $a=2$.
\end{thm}

The constant $C(q,a)$ is the one expected, i.e. such that
$\sum_{\substack{n\le x,\\ n\equiv a[q]}}\frac{\Lambda(n)}{n}
-\frac{\log x}{\varphi(q)}-C(q,a)$ goes to
zero when $x$ goes to infinity.

\par 
  When $q$ belongs to "Rumely's list", i.e. in one of the
following set:
\begin{itemize}
    \item \,\,$\{k\le 72\}$

    \item \,\,$\{k\le 112, \text{$k$ non premier}\}$

    \item \,$\begin{aligned}\{116, 117, &120, 121, 124, 125, 128, 132, 140,
     143, 144, 156, 163, \\ &169, 180, 216, 243, 256, 360, 420, 432\}\end{aligned}$

\end{itemize}
Theorem 2 of
\cite{Ramare*02}
gives the following.
\begin{thm}{Theorem (2002)}

  When $q$ belongs to Rumely's list and $a$ is prime to $q$, we have
  $\displaystyle
  \sum_{\substack{n\le x,\\ n\equiv a[q]}}\frac{\Lambda(n)}{n}
  =\frac{\log x}{\varphi(q)}+C(q,a)+O^*(1)
  $
  as soon as $x\ge1$.
\end{thm}

More precise bounds are given.




\section{Least prime verifying a condition}


\cite{Bach-Sorenson*96},
\cite{Kadiri*05-2},







  
\begin{flushright}\small\sl{}   Last updated on June 3rd, 2022, by Olivier Ramar\'e
 \end{flushright}
















